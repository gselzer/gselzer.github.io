\documentclass[letterpaper,10pt]{article}

\usepackage[style=phys, sorting=ydnt, maxbibnames=3]{biblatex}  % or style=apa, ieee, etc.
\addbibresource{publications.bib}
\usepackage{latexsym}
\usepackage[empty]{fullpage}
\usepackage{titlesec}
\usepackage{marvosym}
\usepackage[usenames,dvipsnames]{color}
\usepackage{verbatim}
\usepackage{enumitem}
\usepackage{fancyhdr}
\usepackage[english]{babel}
\usepackage{tabularx}
% \usepackage[-1]{pagesel}% http://ctan.org/pkg/pagesel
\input{glyphtounicode}

% Tell biblatex to avoid the entry numbers
\DeclareFieldFormat{labelnumberwidth}{}
\setlength{\biblabelsep}{0pt}

% Font options
\usepackage[sfdefault]{roboto}  % Sans-serif font

\pagestyle{fancy}
\fancyhf{}
\fancyfoot{}
\renewcommand{\headrulewidth}{0pt}
\renewcommand{\footrulewidth}{0pt}

\addtolength{\oddsidemargin}{-0.5in}
\addtolength{\evensidemargin}{-0.5in}
\addtolength{\textwidth}{1in}
\addtolength{\topmargin}{-.5in}
\addtolength{\textheight}{1.0in}

\urlstyle{same}
\raggedbottom
\raggedright
\setlength{\tabcolsep}{0in}

% Section formatting
\titleformat{\section}{\Large\bfseries\scshape\raggedright}{}{0em}{}[\titlerule]

% Ensure PDF is machine readable
\pdfgentounicode=1

% Custom commands
\newcommand{\resumeItem}[1]{\item\small{#1}}
\newcommand{\resumeSubheading}[4]{
\vspace{-1pt}\item
  \begin{tabular*}{0.97\textwidth}[t]{l@{\extracolsep{\fill}}r}
    \textbf{#1} & #2 \\
    \textit{#3} & \textit{#4} \\
  \end{tabular*}\vspace{-7pt}
}
\renewcommand\labelitemii{$\vcenter{\hbox{\tiny$\bullet$}}$}
\newcommand{\resumeSubHeadingList}{\begin{itemize}[leftmargin=0.15in, label={}]}
\newcommand{\resumeSubHeadingListEnd}{\end{itemize}}

% Hyperlink setup (moved after fancyhdr to address warning)
\usepackage[hidelinks]{hyperref}
\definecolor{darkblue}{RGB}{0,0,139}
\hypersetup{
    colorlinks=true,
    linkcolor=darkblue,
    filecolor=darkblue,
    urlcolor=darkblue,
}

\begin{document}

\begin{center}
  \textbf{\Huge Gabriel Selzer} \\
  \small (608) 509-5506 $|$ \href{mailto:gabrieljamesselzer@gmail.com}{gabrieljamesselzer@gmail.com} $|$ 
  \href{https://gselzer.github.io}{gselzer.github.io}
\end{center}

% \section*{Summary}
% Software engineer with a strong background in software architecture (Java, Python) and bioimage analysis.


\section{Experience}
\resumeSubHeadingList
  \resumeSubheading
      {Software Engineer}{May 2024 -- Present}
      {Eliceiri Lab (LOCI), University of Wisconsin-Madison}{Madison, WI}
      \resumeSubHeadingList
          \resumeItem{\textbullet\ Collaborated with Harvard Medical School to extract \href{https://pyapp-kit.github.io/ndv/latest/}{ndv} from internal component into a reusable standalone library}
          \resumeItem{\textbullet\ Built histogram visualization for real-time microscope tuning based on user research with imaging scientists}
          \resumeItem{\textbullet\ Developed GitHub Actions CI/CD pipelines, automating testing and documentation deployment with 80\%+ coverage requirements}
      \resumeSubHeadingListEnd
  \resumeSubheading
      {(Graduate) Research Assistant}{Aug 2017 -- May 2024}
      {Eliceiri Lab (LOCI), University of Wisconsin-Madison}{Madison, WI}
      \resumeSubHeadingList
          \resumeItem{\textbullet\ Led development of napari-imagej in collaboration with Chan Zuckerberg Initiative, enabling napari users to access Fiji's tools without Java expertise}
          \resumeItem{\textbullet\ Architected SciJava Ops declarative algorithms framework, targeted for inclusion in Fiji core, reaching thousands of daily users}
      \resumeSubHeadingListEnd
\resumeSubHeadingListEnd

\section{Projects}
\resumeSubHeadingList
  \resumeSubheading
      {\href{https://pyapp-kit.github.io/ndv/latest/}{ndv}}{Jun 2024 -- Present}
      {n-dimensional data viewer}{Python, VisPy, Qt}
      \resumeSubHeadingList
          \resumeItem{\textbullet\ Co-developed an MVC architecture enabling usage from PyQt, Jupyter, and WxPython backed by VisPy or pygfx}
          \resumeItem{\textbullet\ Implemented asynchronous, multi-channel histogram visualization, handling complex, remote, and/or large datasets}
          \resumeItem{\textbullet\ \textbf{Impact: 81 Github stars, approximately 1000 downloads per month, central component of \href{https://github.com/pymmcore-plus/pymmcore-gui}{pymmcore-gui}}}
      \resumeSubHeadingListEnd
  \resumeSubheading
      {\href{https://napari.imagej.net/en/stable/}{napari-imagej}}{Dec 2021 -- Aug 2023}
      {Interoperable user interface bridging napari and Fiji/ImageJ}{Python, Java, napari, Fiji}
      \resumeSubHeadingList
        \resumeItem{\textbullet\ Leveraged zero-copy data conversions between ImageJ and NumPy for high-performance interoperability}
        \resumeItem{\textbullet\ Engineered asynchronous ImageJ2 initialization with Qt QThreads to avoid seconds to minutes of UI blocking}
        \resumeItem{\textbullet\ Implemented automatic UI generation for Fiji plugins, enabling invocation as if they were native Python functions}
        \resumeItem{\textbullet\ \textbf{Impact: 31 Github stars, approximately 80 downloads per month, communications paper published in Nature Methods}}
      \resumeSubHeadingListEnd
\resumeSubheading
      {\href{https://ops.scijava.org/en/latest/}{SciJava Ops}}{Apr 2018 -- Sep 2024}
      {Declarative Image Processing for Fiji/ImageJ}{Java 11, Fiji, OpenCV}
      \resumeSubHeadingList
        \resumeItem{\textbullet\ Designed declarative algorithm discovery and invocation across multiple libraries (ImageJ, OpenCV, NumPy)}
        \resumeItem{\textbullet\ Implemented type-based algorithm routing with automatic data conversion, enabling data-structure-agnostic invocation}
        \resumeItem{\textbullet\ \textbf{Impact: Slated for inclusion in Core Fiji, Paper published in Frontiers in Bioinformatics}}
      \resumeSubHeadingListEnd
\resumeSubHeadingListEnd

\section{Education}
\resumeSubHeadingList
\vspace{-1pt}\item
  \begin{tabular*}{0.97\textwidth}[t]{l@{\extracolsep{\fill}}r}
    \textbf{University of Wisconsin, Madison} & Madison, WI \\
    \textit{Computer Science, M.S.} & \textit{May 2024} \\
    \textit{Electrical Engineering, B.S.} & \textit{Dec 2021} \\
    \textit{Computer Science, B.S.} & \textit{Dec 2021} \\
  \end{tabular*}\vspace{-7pt}
\resumeSubHeadingListEnd

\section{Publications}
\resumeSubHeadingList
\vspace{-1pt}\item
% If/when more papers are listed, may be good to change this to "Selected Publications"
% and cite just the "best" ones.
% The command below would cite a particular paper
% \nocite{Selzer_Rueden_Hiner_Evans_Harrington_Eliceiri_2023}
% This one cites them all
\nocite{*}
\printbibliography[heading=none]
\resumeSubHeadingListEnd

\section{Technical Skills}
\resumeSubHeadingList
  \resumeItem{\textbf{Professional}: Python, Java, Git/GitHub, IntelliJ, VSCode, Qt, Unix, Jupyter, GitHub Actions, JUnit 4/5, pytest, Maven, setuptools}
  \resumeItem{\textbf{Academic/Hobbyist}: Tensorflow, PyTorch, Rust, C/C++, WebGPU, Javascript/HTML/CSS, CUDA}
\resumeSubHeadingListEnd


\end{document}